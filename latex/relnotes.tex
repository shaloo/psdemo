%% Generated by Sphinx.
\def\sphinxdocclass{report}
\documentclass[letterpaper,10pt,english]{sphinxmanual}
\ifdefined\pdfpxdimen
   \let\sphinxpxdimen\pdfpxdimen\else\newdimen\sphinxpxdimen
\fi \sphinxpxdimen=.75bp\relax

\PassOptionsToPackage{warn}{textcomp}
\usepackage[utf8]{inputenc}
\ifdefined\DeclareUnicodeCharacter
% support both utf8 and utf8x syntaxes
\edef\sphinxdqmaybe{\ifdefined\DeclareUnicodeCharacterAsOptional\string"\fi}
  \DeclareUnicodeCharacter{\sphinxdqmaybe00A0}{\nobreakspace}
  \DeclareUnicodeCharacter{\sphinxdqmaybe2500}{\sphinxunichar{2500}}
  \DeclareUnicodeCharacter{\sphinxdqmaybe2502}{\sphinxunichar{2502}}
  \DeclareUnicodeCharacter{\sphinxdqmaybe2514}{\sphinxunichar{2514}}
  \DeclareUnicodeCharacter{\sphinxdqmaybe251C}{\sphinxunichar{251C}}
  \DeclareUnicodeCharacter{\sphinxdqmaybe2572}{\textbackslash}
\fi
\usepackage{cmap}
\usepackage[T1]{fontenc}
\usepackage{amsmath,amssymb,amstext}
\usepackage{babel}
\usepackage{times}
\usepackage[Bjarne]{fncychap}
\usepackage{sphinx}

\fvset{fontsize=\small}
\usepackage{geometry}

% Include hyperref last.
\usepackage{hyperref}
% Fix anchor placement for figures with captions.
\usepackage{hypcap}% it must be loaded after hyperref.
% Set up styles of URL: it should be placed after hyperref.
\urlstyle{same}

\addto\captionsenglish{\renewcommand{\figurename}{Fig.\@ }}
\makeatletter
\def\fnum@figure{\figurename\thefigure{}}
\makeatother
\addto\captionsenglish{\renewcommand{\tablename}{Table }}
\makeatletter
\def\fnum@table{\tablename\thetable{}}
\makeatother
\addto\captionsenglish{\renewcommand{\literalblockname}{Listing}}

\addto\captionsenglish{\renewcommand{\literalblockcontinuedname}{continued from previous page}}
\addto\captionsenglish{\renewcommand{\literalblockcontinuesname}{continues on next page}}
\addto\captionsenglish{\renewcommand{\sphinxnonalphabeticalgroupname}{Non-alphabetical}}
\addto\captionsenglish{\renewcommand{\sphinxsymbolsname}{Symbols}}
\addto\captionsenglish{\renewcommand{\sphinxnumbersname}{Numbers}}

\addto\extrasenglish{\def\pageautorefname{page}}




    \usepackage{sphinx}
    \makeatletter
    \newcommand{\sphinxbackoftitlepage}{%
    \vspace*{\fill}
    \begingroup
      \mbox{}
      \vfill    
      Pulse Secure, LLC
      \newline
      2700 Zanker Road, Suite 200
      \newline
      San Jose, CA 95134 
      \newline
      https://www.pulsesecure.net
      \vspace{2cm}
      \newline
      \vspace{3cm}
      © 2019 by Pulse Secure, LLC. All rights reserved

    \endgroup
    \vspace*{\fill}
    \clearpage
    }
    \makeatother
    

\title{PSdemo Release Notes}
\date{Oct 11, 2019}
\release{(alpha)}
\author{Pulse Secure, LLC}
\newcommand{\sphinxlogo}{\sphinxincludegraphics{psdemo_logo.png}\par}
\renewcommand{\releasename}{1.0}
\makeindex
\begin{document}

\pagestyle{empty}
\sphinxmaketitle
\pagestyle{plain}
\sphinxtableofcontents
\pagestyle{normal}
\phantomsection\label{\detokenize{relnotes::doc}}


\begin{sphinxadmonition}{warning}{Warning:}
This is a sample release note only for demonstration purposes.
\end{sphinxadmonition}

\sphinxstylestrong{Version 1.0}


\chapter{Introduction}
\label{\detokenize{relnotes:introduction}}
This document is the release notes for the product PSdemo Release 9.1R2. This document contains information about what is included in this software release: supported features, feature changes, unsupported features, and known issues. If the information in the release notes differs from the information found in the documentation set, follow the release notes.


\chapter{Hardware Platforms}
\label{\detokenize{relnotes:hardware-platforms}}
You can install and use this software version on the following hardware platforms: • PSA300, PSA3000, PSA5000, PSA7000f, PSA7000c

To download software for these hardware platforms, go to: \sphinxurl{https://support.pulsesecure.net/} Virtual Appliance Editions
This software version is available for the following virtual appliance editions: • Virtual Pulse Secure Appliance (PSA-V)

\begin{sphinxadmonition}{note}{Note:}
From 9.1R1 release onwards, VA-DTE is not supported.

From 9.0R1 release, Pulse Secure has begun the End-of-Life (EOL) process for the VA-SPE virtual appliance. In its place, Pulse Secure has launched the new PSA-V series of virtual appliances designed for use in the data center or with cloud services such as Microsoft Azure and Amazon AWS.
\end{sphinxadmonition}

The following table lists the virtual appliance systems qualified with this release.


\section{Platform Qualified System}
\label{\detokenize{relnotes:platform-qualified-system}}
\sphinxstylestrong{VMware}
\begin{itemize}
\item {} 
HP ProLiant DL380 G5 with Intel(R) Xeon(R) CPU

\item {} 
ESXi 6.7

\item {} 
CentOS 6.6 with Kernel cst-kvm 2.6.32-504.el6.x86\_64

\item {} 
QEMU/KVM v1.4.0

\end{itemize}

\sphinxstylestrong{Linux Server Release 6.4 on an Intel Xeon CPU L5640 @ 2.27GHz}
\begin{itemize}
\item {} 
24GB memory in host

\item {} 
Allocation for virtual appliance ( KVM, Hyper-V)
\begin{itemize}
\item {} 
4vCPU,

\item {} 
4GB memory,

\item {} 
40GB disk space

\end{itemize}

\item {} 
Standard DS2 V2 (2 Core, 2 NICs)

\end{itemize}



\renewcommand{\indexname}{Index}
\printindex
\end{document}