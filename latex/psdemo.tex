%% Generated by Sphinx.
\def\sphinxdocclass{report}
\documentclass[letterpaper,10pt,english]{sphinxmanual}
\ifdefined\pdfpxdimen
   \let\sphinxpxdimen\pdfpxdimen\else\newdimen\sphinxpxdimen
\fi \sphinxpxdimen=.75bp\relax

\PassOptionsToPackage{warn}{textcomp}
\usepackage[utf8]{inputenc}
\ifdefined\DeclareUnicodeCharacter
% support both utf8 and utf8x syntaxes
\edef\sphinxdqmaybe{\ifdefined\DeclareUnicodeCharacterAsOptional\string"\fi}
  \DeclareUnicodeCharacter{\sphinxdqmaybe00A0}{\nobreakspace}
  \DeclareUnicodeCharacter{\sphinxdqmaybe2500}{\sphinxunichar{2500}}
  \DeclareUnicodeCharacter{\sphinxdqmaybe2502}{\sphinxunichar{2502}}
  \DeclareUnicodeCharacter{\sphinxdqmaybe2514}{\sphinxunichar{2514}}
  \DeclareUnicodeCharacter{\sphinxdqmaybe251C}{\sphinxunichar{251C}}
  \DeclareUnicodeCharacter{\sphinxdqmaybe2572}{\textbackslash}
\fi
\usepackage{cmap}
\usepackage[T1]{fontenc}
\usepackage{amsmath,amssymb,amstext}
\usepackage{babel}
\usepackage{times}
\usepackage[Bjarne]{fncychap}
\usepackage{sphinx}

\fvset{fontsize=\small}
\usepackage{geometry}

% Include hyperref last.
\usepackage{hyperref}
% Fix anchor placement for figures with captions.
\usepackage{hypcap}% it must be loaded after hyperref.
% Set up styles of URL: it should be placed after hyperref.
\urlstyle{same}
\addto\captionsenglish{\renewcommand{\contentsname}{Contents:}}

\addto\captionsenglish{\renewcommand{\figurename}{Fig.\@ }}
\makeatletter
\def\fnum@figure{\figurename\thefigure{}}
\makeatother
\addto\captionsenglish{\renewcommand{\tablename}{Table }}
\makeatletter
\def\fnum@table{\tablename\thetable{}}
\makeatother
\addto\captionsenglish{\renewcommand{\literalblockname}{Listing}}

\addto\captionsenglish{\renewcommand{\literalblockcontinuedname}{continued from previous page}}
\addto\captionsenglish{\renewcommand{\literalblockcontinuesname}{continues on next page}}
\addto\captionsenglish{\renewcommand{\sphinxnonalphabeticalgroupname}{Non-alphabetical}}
\addto\captionsenglish{\renewcommand{\sphinxsymbolsname}{Symbols}}
\addto\captionsenglish{\renewcommand{\sphinxnumbersname}{Numbers}}

\addto\extrasenglish{\def\pageautorefname{page}}

\setcounter{tocdepth}{1}


    \usepackage{sphinx}
    \makeatletter
    \newcommand{\sphinxbackoftitlepage}{%
    \vspace*{\fill}
    \begingroup
      \mbox{}
      \vfill    
      Pulse Secure, LLC
      \newline
      2700 Zanker Road, Suite 200
      \newline
      San Jose, CA 95134 
      \newline
      https://www.pulsesecure.net
      \vspace{2cm}
      \newline
      \vspace{3cm}
      © 2019 by Pulse Secure, LLC. All rights reserved

    \endgroup
    \vspace*{\fill}
    \clearpage
    }
    \makeatother
    

\title{PS Demo Documentation}
\date{Oct 11, 2019}
\release{(alpha)}
\author{}
\newcommand{\sphinxlogo}{\sphinxincludegraphics{psdemo_logo.png}\par}
\renewcommand{\releasename}{1.0}
\makeindex
\begin{document}

\pagestyle{empty}
\sphinxmaketitle
\pagestyle{plain}
\sphinxtableofcontents
\pagestyle{normal}
\phantomsection\label{\detokenize{index::doc}}



\chapter{PSdemo Release Notes}
\label{\detokenize{relnotes:product-name-release-notes}}\label{\detokenize{relnotes:doc-relnotes}}\label{\detokenize{relnotes::doc}}
\begin{sphinxadmonition}{warning}{Warning:}
This is a sample release note only for demonstration purposes.
\end{sphinxadmonition}

\sphinxstylestrong{Version 1.0}


\section{Introduction}
\label{\detokenize{relnotes:introduction}}
This document is the release notes for the product PSdemo Release 9.1R2. This document contains information about what is included in this software release: supported features, feature changes, unsupported features, and known issues. If the information in the release notes differs from the information found in the documentation set, follow the release notes.


\section{Hardware Platforms}
\label{\detokenize{relnotes:hardware-platforms}}
You can install and use this software version on the following hardware platforms: • PSA300, PSA3000, PSA5000, PSA7000f, PSA7000c

To download software for these hardware platforms, go to: \sphinxurl{https://support.pulsesecure.net/} Virtual Appliance Editions
This software version is available for the following virtual appliance editions: • Virtual Pulse Secure Appliance (PSA-V)

\begin{sphinxadmonition}{note}{Note:}
From 9.1R1 release onwards, VA-DTE is not supported.

From 9.0R1 release, Pulse Secure has begun the End-of-Life (EOL) process for the VA-SPE virtual appliance. In its place, Pulse Secure has launched the new PSA-V series of virtual appliances designed for use in the data center or with cloud services such as Microsoft Azure and Amazon AWS.
\end{sphinxadmonition}

The following table lists the virtual appliance systems qualified with this release.


\subsection{Platform Qualified System}
\label{\detokenize{relnotes:platform-qualified-system}}
\sphinxstylestrong{VMware}
\begin{itemize}
\item {} 
HP ProLiant DL380 G5 with Intel(R) Xeon(R) CPU

\item {} 
ESXi 6.7

\item {} 
CentOS 6.6 with Kernel cst-kvm 2.6.32-504.el6.x86\_64

\item {} 
QEMU/KVM v1.4.0

\end{itemize}

\sphinxstylestrong{Linux Server Release 6.4 on an Intel Xeon CPU L5640 @ 2.27GHz}
\begin{itemize}
\item {} 
24GB memory in host

\item {} 
Allocation for virtual appliance ( KVM, Hyper-V)
\begin{itemize}
\item {} 
4vCPU,

\item {} 
4GB memory,

\item {} 
40GB disk space

\end{itemize}

\item {} 
Standard DS2 V2 (2 Core, 2 NICs)

\end{itemize}


\chapter{PSdemo Getting Started Guide}
\label{\detokenize{get-started:product-name-getting-started-guide}}\label{\detokenize{get-started:doc-get-started}}\label{\detokenize{get-started::doc}}
TBD

\begin{sphinxShadowBox}
\begin{itemize}
\item {} 
\phantomsection\label{\detokenize{get-started:id1}}{\hyperref[\detokenize{get-started:introduction}]{\sphinxcrossref{Introduction}}}

\item {} 
\phantomsection\label{\detokenize{get-started:id2}}{\hyperref[\detokenize{get-started:chapter-1}]{\sphinxcrossref{Chapter 1}}}

\item {} 
\phantomsection\label{\detokenize{get-started:id3}}{\hyperref[\detokenize{get-started:chapter-2}]{\sphinxcrossref{Chapter 2}}}

\end{itemize}
\end{sphinxShadowBox}


\section{Introduction}
\label{\detokenize{get-started:introduction}}
TBD


\section{Chapter 1}
\label{\detokenize{get-started:chapter-1}}
TBD


\section{Chapter 2}
\label{\detokenize{get-started:chapter-2}}

\chapter{PSdemo Developer’s Guide}
\label{\detokenize{dev-guide:product-name-developer-s-guide}}\label{\detokenize{dev-guide:doc-dev-guide}}\label{\detokenize{dev-guide::doc}}
TBD

\begin{sphinxShadowBox}
\begin{itemize}
\item {} 
\phantomsection\label{\detokenize{dev-guide:id1}}{\hyperref[\detokenize{dev-guide:introduction}]{\sphinxcrossref{Introduction}}}

\item {} 
\phantomsection\label{\detokenize{dev-guide:id2}}{\hyperref[\detokenize{dev-guide:api-for-some-functionality}]{\sphinxcrossref{API for some functionality}}}
\begin{itemize}
\item {} 
\phantomsection\label{\detokenize{dev-guide:id3}}{\hyperref[\detokenize{dev-guide:pet-store-apis}]{\sphinxcrossref{Pet Store APIs}}}
\begin{itemize}
\item {} 
\phantomsection\label{\detokenize{dev-guide:id4}}{\hyperref[\detokenize{dev-guide:pet}]{\sphinxcrossref{pet}}}

\item {} 
\phantomsection\label{\detokenize{dev-guide:id5}}{\hyperref[\detokenize{dev-guide:store}]{\sphinxcrossref{store}}}

\item {} 
\phantomsection\label{\detokenize{dev-guide:id6}}{\hyperref[\detokenize{dev-guide:user}]{\sphinxcrossref{user}}}

\end{itemize}

\end{itemize}

\item {} 
\phantomsection\label{\detokenize{dev-guide:id7}}{\hyperref[\detokenize{dev-guide:example-code}]{\sphinxcrossref{Example Code}}}
\begin{itemize}
\item {} 
\phantomsection\label{\detokenize{dev-guide:id8}}{\hyperref[\detokenize{dev-guide:using-api-for-xyz-use-case}]{\sphinxcrossref{Using API for xyz use case}}}

\item {} 
\phantomsection\label{\detokenize{dev-guide:id9}}{\hyperref[\detokenize{dev-guide:using-api-for-mno-use-case}]{\sphinxcrossref{Using API for mno use case}}}

\end{itemize}

\end{itemize}
\end{sphinxShadowBox}


\bigskip\hrule\bigskip



\section{Introduction}
\label{\detokenize{dev-guide:introduction}}
TBD


\section{API for some functionality}
\label{\detokenize{dev-guide:api-for-some-functionality}}
TBD


\subsection{Pet Store APIs}
\label{\detokenize{dev-guide:pet-store-apis}}

\subsubsection{pet}
\label{\detokenize{dev-guide:pet}}
\begin{sphinxadmonition}{note}{POST /pet}

Add a new pet to the store\begin{itemize}
\item {} 
\sphinxstylestrong{Description: 
}

\item {} 
\sphinxstylestrong{Consumes: 
}{[}‘application/json’, ‘application/xml’{]}

\item {} 
\sphinxstylestrong{Produces: 
}{[}‘application/xml’, ‘application/json’{]}

\end{itemize}


\sphinxstylestrong{Parameters}


\begin{savenotes}\sphinxattablestart
\centering
\begin{tabulary}{\linewidth}[t]{|T|T|T|T|}
\hline
\sphinxstyletheadfamily 
Name
&\sphinxstyletheadfamily 
Position
&\sphinxstyletheadfamily 
Description
&\sphinxstyletheadfamily 
Type
\\
\hline
body
&
body
&
Pet object that needs to be added to the store
&

\\
\hline
\end{tabulary}
\par
\sphinxattableend\end{savenotes}

\sphinxstylestrong{Responses}

\sphinxstyleemphasis{405 - Invalid input}
\end{sphinxadmonition}

\begin{sphinxadmonition}{note}{PUT /pet}

Update an existing pet\begin{itemize}
\item {} 
\sphinxstylestrong{Description: 
}

\item {} 
\sphinxstylestrong{Consumes: 
}{[}‘application/json’, ‘application/xml’{]}

\item {} 
\sphinxstylestrong{Produces: 
}{[}‘application/xml’, ‘application/json’{]}

\end{itemize}


\sphinxstylestrong{Parameters}


\begin{savenotes}\sphinxattablestart
\centering
\begin{tabulary}{\linewidth}[t]{|T|T|T|T|}
\hline
\sphinxstyletheadfamily 
Name
&\sphinxstyletheadfamily 
Position
&\sphinxstyletheadfamily 
Description
&\sphinxstyletheadfamily 
Type
\\
\hline
body
&
body
&
Pet object that needs to be added to the store
&

\\
\hline
\end{tabulary}
\par
\sphinxattableend\end{savenotes}

\sphinxstylestrong{Responses}

\sphinxstyleemphasis{400 - Invalid ID supplied}

\sphinxstyleemphasis{404 - Pet not found}

\sphinxstyleemphasis{405 - Validation exception}
\end{sphinxadmonition}

\begin{sphinxadmonition}{note}{GET /pet/findByStatus}

Finds Pets by status\begin{itemize}
\item {} 
\sphinxstylestrong{Description: 
}Multiple status values can be provided with comma separated strings

\item {} 
\sphinxstylestrong{Produces: 
}{[}‘application/xml’, ‘application/json’{]}

\end{itemize}


\sphinxstylestrong{Parameters}


\begin{savenotes}\sphinxattablestart
\centering
\begin{tabulary}{\linewidth}[t]{|T|T|T|T|}
\hline
\sphinxstyletheadfamily 
Name
&\sphinxstyletheadfamily 
Position
&\sphinxstyletheadfamily 
Description
&\sphinxstyletheadfamily 
Type
\\
\hline
status
&
query
&
Status values that need to be considered for filter
&
array
\\
\hline
\end{tabulary}
\par
\sphinxattableend\end{savenotes}

\sphinxstylestrong{Responses}

\sphinxstyleemphasis{200 - successful operation}

\sphinxstyleemphasis{400 - Invalid status value}
\end{sphinxadmonition}

\begin{sphinxadmonition}{note}{GET /pet/findByTags}

Finds Pets by tags\begin{itemize}
\item {} 
\sphinxstylestrong{Description: 
}Multiple tags can be provided with comma separated strings. Use tag1, tag2, tag3 for testing.

\item {} 
\sphinxstylestrong{Produces: 
}{[}‘application/xml’, ‘application/json’{]}

\end{itemize}


\sphinxstylestrong{Parameters}


\begin{savenotes}\sphinxattablestart
\centering
\begin{tabulary}{\linewidth}[t]{|T|T|T|T|}
\hline
\sphinxstyletheadfamily 
Name
&\sphinxstyletheadfamily 
Position
&\sphinxstyletheadfamily 
Description
&\sphinxstyletheadfamily 
Type
\\
\hline
tags
&
query
&
Tags to filter by
&
array
\\
\hline
\end{tabulary}
\par
\sphinxattableend\end{savenotes}

\sphinxstylestrong{Responses}

\sphinxstyleemphasis{200 - successful operation}

\sphinxstyleemphasis{400 - Invalid tag value}
\end{sphinxadmonition}

\begin{sphinxadmonition}{note}{GET /pet/\{petId\}}

Find pet by ID\begin{itemize}
\item {} 
\sphinxstylestrong{Description: 
}Returns a single pet

\item {} 
\sphinxstylestrong{Produces: 
}{[}‘application/xml’, ‘application/json’{]}

\end{itemize}


\sphinxstylestrong{Parameters}


\begin{savenotes}\sphinxattablestart
\centering
\begin{tabulary}{\linewidth}[t]{|T|T|T|T|}
\hline
\sphinxstyletheadfamily 
Name
&\sphinxstyletheadfamily 
Position
&\sphinxstyletheadfamily 
Description
&\sphinxstyletheadfamily 
Type
\\
\hline
petId
&
path
&
ID of pet to return
&
integer
\\
\hline
\end{tabulary}
\par
\sphinxattableend\end{savenotes}

\sphinxstylestrong{Responses}

\sphinxstyleemphasis{200 - successful operation}

\sphinxstyleemphasis{400 - Invalid ID supplied}

\sphinxstyleemphasis{404 - Pet not found}
\end{sphinxadmonition}

\begin{sphinxadmonition}{note}{POST /pet/\{petId\}}

Updates a pet in the store with form data\begin{itemize}
\item {} 
\sphinxstylestrong{Description: 
}

\item {} 
\sphinxstylestrong{Consumes: 
}{[}‘application/x-www-form-urlencoded’{]}

\item {} 
\sphinxstylestrong{Produces: 
}{[}‘application/xml’, ‘application/json’{]}

\end{itemize}


\sphinxstylestrong{Parameters}


\begin{savenotes}\sphinxattablestart
\centering
\begin{tabulary}{\linewidth}[t]{|T|T|T|T|}
\hline
\sphinxstyletheadfamily 
Name
&\sphinxstyletheadfamily 
Position
&\sphinxstyletheadfamily 
Description
&\sphinxstyletheadfamily 
Type
\\
\hline
petId
&
path
&
ID of pet that needs to be updated
&
integer
\\
\hline
name
&
formData
&
Updated name of the pet
&
string
\\
\hline
status
&
formData
&
Updated status of the pet
&
string
\\
\hline
\end{tabulary}
\par
\sphinxattableend\end{savenotes}

\sphinxstylestrong{Responses}

\sphinxstyleemphasis{405 - Invalid input}
\end{sphinxadmonition}

\begin{sphinxadmonition}{note}{DELETE /pet/\{petId\}}

Deletes a pet\begin{itemize}
\item {} 
\sphinxstylestrong{Description: 
}

\item {} 
\sphinxstylestrong{Produces: 
}{[}‘application/xml’, ‘application/json’{]}

\end{itemize}


\sphinxstylestrong{Parameters}


\begin{savenotes}\sphinxattablestart
\centering
\begin{tabulary}{\linewidth}[t]{|T|T|T|T|}
\hline
\sphinxstyletheadfamily 
Name
&\sphinxstyletheadfamily 
Position
&\sphinxstyletheadfamily 
Description
&\sphinxstyletheadfamily 
Type
\\
\hline
api\_key
&
header
&

&
string
\\
\hline
petId
&
path
&
Pet id to delete
&
integer
\\
\hline
\end{tabulary}
\par
\sphinxattableend\end{savenotes}

\sphinxstylestrong{Responses}

\sphinxstyleemphasis{400 - Invalid ID supplied}

\sphinxstyleemphasis{404 - Pet not found}
\end{sphinxadmonition}

\begin{sphinxadmonition}{note}{POST /pet/\{petId\}/uploadImage}

uploads an image\begin{itemize}
\item {} 
\sphinxstylestrong{Description: 
}

\item {} 
\sphinxstylestrong{Consumes: 
}{[}‘multipart/form-data’{]}

\item {} 
\sphinxstylestrong{Produces: 
}{[}‘application/json’{]}

\end{itemize}


\sphinxstylestrong{Parameters}


\begin{savenotes}\sphinxattablestart
\centering
\begin{tabulary}{\linewidth}[t]{|T|T|T|T|}
\hline
\sphinxstyletheadfamily 
Name
&\sphinxstyletheadfamily 
Position
&\sphinxstyletheadfamily 
Description
&\sphinxstyletheadfamily 
Type
\\
\hline
petId
&
path
&
ID of pet to update
&
integer
\\
\hline
additionalMetadata
&
formData
&
Additional data to pass to server
&
string
\\
\hline
file
&
formData
&
file to upload
&
file
\\
\hline
\end{tabulary}
\par
\sphinxattableend\end{savenotes}

\sphinxstylestrong{Responses}

\sphinxstyleemphasis{200 - successful operation}
\end{sphinxadmonition}


\subsubsection{store}
\label{\detokenize{dev-guide:store}}
\begin{sphinxadmonition}{note}{GET /store/inventory}

Returns pet inventories by status\begin{itemize}
\item {} 
\sphinxstylestrong{Description: 
}Returns a map of status codes to quantities

\item {} 
\sphinxstylestrong{Produces: 
}{[}‘application/json’{]}

\end{itemize}


\sphinxstylestrong{Parameters}


\begin{savenotes}\sphinxattablestart
\centering
\begin{tabulary}{\linewidth}[t]{|T|T|T|T|}
\hline
\sphinxstyletheadfamily 
Name
&\sphinxstyletheadfamily 
Position
&\sphinxstyletheadfamily 
Description
&\sphinxstyletheadfamily 
Type
\\
\hline
\end{tabulary}
\par
\sphinxattableend\end{savenotes}

\sphinxstylestrong{Responses}

\sphinxstyleemphasis{200 - successful operation}
\end{sphinxadmonition}

\begin{sphinxadmonition}{note}{POST /store/order}

Place an order for a pet\begin{itemize}
\item {} 
\sphinxstylestrong{Description: 
}

\item {} 
\sphinxstylestrong{Produces: 
}{[}‘application/xml’, ‘application/json’{]}

\end{itemize}


\sphinxstylestrong{Parameters}


\begin{savenotes}\sphinxattablestart
\centering
\begin{tabulary}{\linewidth}[t]{|T|T|T|T|}
\hline
\sphinxstyletheadfamily 
Name
&\sphinxstyletheadfamily 
Position
&\sphinxstyletheadfamily 
Description
&\sphinxstyletheadfamily 
Type
\\
\hline
body
&
body
&
order placed for purchasing the pet
&

\\
\hline
\end{tabulary}
\par
\sphinxattableend\end{savenotes}

\sphinxstylestrong{Responses}

\sphinxstyleemphasis{200 - successful operation}

\sphinxstyleemphasis{400 - Invalid Order}
\end{sphinxadmonition}

\begin{sphinxadmonition}{note}{GET /store/order/\{orderId\}}

Find purchase order by ID\begin{itemize}
\item {} 
\sphinxstylestrong{Description: 
}For valid response try integer IDs with value \textgreater{}= 1 and \textless{}= 10. Other values will generated exceptions

\item {} 
\sphinxstylestrong{Produces: 
}{[}‘application/xml’, ‘application/json’{]}

\end{itemize}


\sphinxstylestrong{Parameters}


\begin{savenotes}\sphinxattablestart
\centering
\begin{tabulary}{\linewidth}[t]{|T|T|T|T|}
\hline
\sphinxstyletheadfamily 
Name
&\sphinxstyletheadfamily 
Position
&\sphinxstyletheadfamily 
Description
&\sphinxstyletheadfamily 
Type
\\
\hline
orderId
&
path
&
ID of pet that needs to be fetched
&
integer
\\
\hline
\end{tabulary}
\par
\sphinxattableend\end{savenotes}

\sphinxstylestrong{Responses}

\sphinxstyleemphasis{200 - successful operation}

\sphinxstyleemphasis{400 - Invalid ID supplied}

\sphinxstyleemphasis{404 - Order not found}
\end{sphinxadmonition}

\begin{sphinxadmonition}{note}{DELETE /store/order/\{orderId\}}

Delete purchase order by ID\begin{itemize}
\item {} 
\sphinxstylestrong{Description: 
}For valid response try integer IDs with positive integer value. Negative or non-integer values will generate API errors

\item {} 
\sphinxstylestrong{Produces: 
}{[}‘application/xml’, ‘application/json’{]}

\end{itemize}


\sphinxstylestrong{Parameters}


\begin{savenotes}\sphinxattablestart
\centering
\begin{tabulary}{\linewidth}[t]{|T|T|T|T|}
\hline
\sphinxstyletheadfamily 
Name
&\sphinxstyletheadfamily 
Position
&\sphinxstyletheadfamily 
Description
&\sphinxstyletheadfamily 
Type
\\
\hline
orderId
&
path
&
ID of the order that needs to be deleted
&
integer
\\
\hline
\end{tabulary}
\par
\sphinxattableend\end{savenotes}

\sphinxstylestrong{Responses}

\sphinxstyleemphasis{400 - Invalid ID supplied}

\sphinxstyleemphasis{404 - Order not found}
\end{sphinxadmonition}


\subsubsection{user}
\label{\detokenize{dev-guide:user}}
\begin{sphinxadmonition}{note}{POST /user}

Create user\begin{itemize}
\item {} 
\sphinxstylestrong{Description: 
}This can only be done by the logged in user.

\item {} 
\sphinxstylestrong{Produces: 
}{[}‘application/xml’, ‘application/json’{]}

\end{itemize}


\sphinxstylestrong{Parameters}


\begin{savenotes}\sphinxattablestart
\centering
\begin{tabulary}{\linewidth}[t]{|T|T|T|T|}
\hline
\sphinxstyletheadfamily 
Name
&\sphinxstyletheadfamily 
Position
&\sphinxstyletheadfamily 
Description
&\sphinxstyletheadfamily 
Type
\\
\hline
body
&
body
&
Created user object
&

\\
\hline
\end{tabulary}
\par
\sphinxattableend\end{savenotes}

\sphinxstylestrong{Responses}

\sphinxstyleemphasis{default - successful operation}
\end{sphinxadmonition}

\begin{sphinxadmonition}{note}{POST /user/createWithArray}

Creates list of users with given input array\begin{itemize}
\item {} 
\sphinxstylestrong{Description: 
}

\item {} 
\sphinxstylestrong{Produces: 
}{[}‘application/xml’, ‘application/json’{]}

\end{itemize}


\sphinxstylestrong{Parameters}


\begin{savenotes}\sphinxattablestart
\centering
\begin{tabulary}{\linewidth}[t]{|T|T|T|T|}
\hline
\sphinxstyletheadfamily 
Name
&\sphinxstyletheadfamily 
Position
&\sphinxstyletheadfamily 
Description
&\sphinxstyletheadfamily 
Type
\\
\hline
body
&
body
&
List of user object
&

\\
\hline
\end{tabulary}
\par
\sphinxattableend\end{savenotes}

\sphinxstylestrong{Responses}

\sphinxstyleemphasis{default - successful operation}
\end{sphinxadmonition}

\begin{sphinxadmonition}{note}{POST /user/createWithList}

Creates list of users with given input array\begin{itemize}
\item {} 
\sphinxstylestrong{Description: 
}

\item {} 
\sphinxstylestrong{Produces: 
}{[}‘application/xml’, ‘application/json’{]}

\end{itemize}


\sphinxstylestrong{Parameters}


\begin{savenotes}\sphinxattablestart
\centering
\begin{tabulary}{\linewidth}[t]{|T|T|T|T|}
\hline
\sphinxstyletheadfamily 
Name
&\sphinxstyletheadfamily 
Position
&\sphinxstyletheadfamily 
Description
&\sphinxstyletheadfamily 
Type
\\
\hline
body
&
body
&
List of user object
&

\\
\hline
\end{tabulary}
\par
\sphinxattableend\end{savenotes}

\sphinxstylestrong{Responses}

\sphinxstyleemphasis{default - successful operation}
\end{sphinxadmonition}

\begin{sphinxadmonition}{note}{GET /user/login}

Logs user into the system\begin{itemize}
\item {} 
\sphinxstylestrong{Description: 
}

\item {} 
\sphinxstylestrong{Produces: 
}{[}‘application/xml’, ‘application/json’{]}

\end{itemize}


\sphinxstylestrong{Parameters}


\begin{savenotes}\sphinxattablestart
\centering
\begin{tabulary}{\linewidth}[t]{|T|T|T|T|}
\hline
\sphinxstyletheadfamily 
Name
&\sphinxstyletheadfamily 
Position
&\sphinxstyletheadfamily 
Description
&\sphinxstyletheadfamily 
Type
\\
\hline
username
&
query
&
The user name for login
&
string
\\
\hline
password
&
query
&
The password for login in clear text
&
string
\\
\hline
\end{tabulary}
\par
\sphinxattableend\end{savenotes}

\sphinxstylestrong{Responses}

\sphinxstyleemphasis{200 - successful operation}

\sphinxstyleemphasis{400 - Invalid username/password supplied}
\end{sphinxadmonition}

\begin{sphinxadmonition}{note}{GET /user/logout}

Logs out current logged in user session\begin{itemize}
\item {} 
\sphinxstylestrong{Description: 
}

\item {} 
\sphinxstylestrong{Produces: 
}{[}‘application/xml’, ‘application/json’{]}

\end{itemize}


\sphinxstylestrong{Parameters}


\begin{savenotes}\sphinxattablestart
\centering
\begin{tabulary}{\linewidth}[t]{|T|T|T|T|}
\hline
\sphinxstyletheadfamily 
Name
&\sphinxstyletheadfamily 
Position
&\sphinxstyletheadfamily 
Description
&\sphinxstyletheadfamily 
Type
\\
\hline
\end{tabulary}
\par
\sphinxattableend\end{savenotes}

\sphinxstylestrong{Responses}

\sphinxstyleemphasis{default - successful operation}
\end{sphinxadmonition}

\begin{sphinxadmonition}{note}{GET /user/\{username\}}

Get user by user name\begin{itemize}
\item {} 
\sphinxstylestrong{Description: 
}

\item {} 
\sphinxstylestrong{Produces: 
}{[}‘application/xml’, ‘application/json’{]}

\end{itemize}


\sphinxstylestrong{Parameters}


\begin{savenotes}\sphinxattablestart
\centering
\begin{tabulary}{\linewidth}[t]{|T|T|T|T|}
\hline
\sphinxstyletheadfamily 
Name
&\sphinxstyletheadfamily 
Position
&\sphinxstyletheadfamily 
Description
&\sphinxstyletheadfamily 
Type
\\
\hline
username
&
path
&
The name that needs to be fetched. Use user1 for testing. 
&
string
\\
\hline
\end{tabulary}
\par
\sphinxattableend\end{savenotes}

\sphinxstylestrong{Responses}

\sphinxstyleemphasis{200 - successful operation}

\sphinxstyleemphasis{400 - Invalid username supplied}

\sphinxstyleemphasis{404 - User not found}
\end{sphinxadmonition}

\begin{sphinxadmonition}{note}{PUT /user/\{username\}}

Updated user\begin{itemize}
\item {} 
\sphinxstylestrong{Description: 
}This can only be done by the logged in user.

\item {} 
\sphinxstylestrong{Produces: 
}{[}‘application/xml’, ‘application/json’{]}

\end{itemize}


\sphinxstylestrong{Parameters}


\begin{savenotes}\sphinxattablestart
\centering
\begin{tabulary}{\linewidth}[t]{|T|T|T|T|}
\hline
\sphinxstyletheadfamily 
Name
&\sphinxstyletheadfamily 
Position
&\sphinxstyletheadfamily 
Description
&\sphinxstyletheadfamily 
Type
\\
\hline
username
&
path
&
name that need to be updated
&
string
\\
\hline
body
&
body
&
Updated user object
&

\\
\hline
\end{tabulary}
\par
\sphinxattableend\end{savenotes}

\sphinxstylestrong{Responses}

\sphinxstyleemphasis{400 - Invalid user supplied}

\sphinxstyleemphasis{404 - User not found}
\end{sphinxadmonition}

\begin{sphinxadmonition}{note}{DELETE /user/\{username\}}

Delete user\begin{itemize}
\item {} 
\sphinxstylestrong{Description: 
}This can only be done by the logged in user.

\item {} 
\sphinxstylestrong{Produces: 
}{[}‘application/xml’, ‘application/json’{]}

\end{itemize}


\sphinxstylestrong{Parameters}


\begin{savenotes}\sphinxattablestart
\centering
\begin{tabulary}{\linewidth}[t]{|T|T|T|T|}
\hline
\sphinxstyletheadfamily 
Name
&\sphinxstyletheadfamily 
Position
&\sphinxstyletheadfamily 
Description
&\sphinxstyletheadfamily 
Type
\\
\hline
username
&
path
&
The name that needs to be deleted
&
string
\\
\hline
\end{tabulary}
\par
\sphinxattableend\end{savenotes}

\sphinxstylestrong{Responses}

\sphinxstyleemphasis{400 - Invalid username supplied}

\sphinxstyleemphasis{404 - User not found}
\end{sphinxadmonition}


\section{Example Code}
\label{\detokenize{dev-guide:example-code}}
Perform the setup using the following commands

\begin{sphinxVerbatim}[commandchars=\\\{\}]
\PYG{n}{bash} \PYG{n}{xyz}\PYG{o}{.}\PYG{n}{py} \PYG{o}{\PYGZhy{}}\PYG{n}{o} \PYG{o}{\PYGZlt{}}\PYG{n}{outputfile}\PYG{o}{\PYGZgt{}} \PYG{o}{\PYGZhy{}}\PYG{n}{d} \PYG{o}{\PYGZlt{}}\PYG{n}{inputdir}\PYG{o}{\PYGZgt{}} \PYG{o}{\PYGZhy{}}\PYG{n}{t} \PYG{o}{\PYGZlt{}}\PYG{n+nb}{type}\PYG{o}{\PYGZgt{}}
\end{sphinxVerbatim}


\subsection{Using API for xyz use case}
\label{\detokenize{dev-guide:using-api-for-xyz-use-case}}

\subsection{Using API for mno use case}
\label{\detokenize{dev-guide:using-api-for-mno-use-case}}

\chapter{PSdemo User Guide}
\label{\detokenize{user-guide:product-name-user-guide}}\label{\detokenize{user-guide:doc-user-guide}}\label{\detokenize{user-guide::doc}}
This is the user guide for Pulse Secure product codenamed PSdemo.

\begin{sphinxShadowBox}
\begin{itemize}
\item {} 
\phantomsection\label{\detokenize{user-guide:id1}}{\hyperref[\detokenize{user-guide:revision-history}]{\sphinxcrossref{Revision History}}}

\item {} 
\phantomsection\label{\detokenize{user-guide:id2}}{\hyperref[\detokenize{user-guide:introduction}]{\sphinxcrossref{Introduction}}}

\item {} 
\phantomsection\label{\detokenize{user-guide:id3}}{\hyperref[\detokenize{user-guide:section-a}]{\sphinxcrossref{Section A}}}
\begin{itemize}
\item {} 
\phantomsection\label{\detokenize{user-guide:id4}}{\hyperref[\detokenize{user-guide:part-1}]{\sphinxcrossref{Part 1}}}

\item {} 
\phantomsection\label{\detokenize{user-guide:id5}}{\hyperref[\detokenize{user-guide:part-2}]{\sphinxcrossref{Part 2}}}

\end{itemize}

\item {} 
\phantomsection\label{\detokenize{user-guide:id6}}{\hyperref[\detokenize{user-guide:section-b}]{\sphinxcrossref{Section B}}}
\begin{itemize}
\item {} 
\phantomsection\label{\detokenize{user-guide:id7}}{\hyperref[\detokenize{user-guide:how-to-do-xyz}]{\sphinxcrossref{How to do xyz}}}

\item {} 
\phantomsection\label{\detokenize{user-guide:id8}}{\hyperref[\detokenize{user-guide:how-to-setup-mno}]{\sphinxcrossref{How to setup mno}}}

\end{itemize}

\end{itemize}
\end{sphinxShadowBox}


\bigskip\hrule\bigskip



\section{Revision History}
\label{\detokenize{user-guide:revision-history}}
TBD


\section{Introduction}
\label{\detokenize{user-guide:introduction}}
TBD


\section{Section A}
\label{\detokenize{user-guide:section-a}}
TBD


\subsection{Part 1}
\label{\detokenize{user-guide:part-1}}

\subsection{Part 2}
\label{\detokenize{user-guide:part-2}}

\section{Section B}
\label{\detokenize{user-guide:section-b}}

\subsection{How to do xyz}
\label{\detokenize{user-guide:how-to-do-xyz}}

\subsection{How to setup mno}
\label{\detokenize{user-guide:how-to-setup-mno}}

\chapter{PSdemo Frequently Asked Questions}
\label{\detokenize{faq:product-name-frequently-asked-questions}}\label{\detokenize{faq:doc-faq}}\label{\detokenize{faq::doc}}
TBD

\begin{sphinxShadowBox}
\begin{itemize}
\item {} 
\phantomsection\label{\detokenize{faq:id1}}{\hyperref[\detokenize{faq:what-is-abc}]{\sphinxcrossref{What is abc?}}}

\item {} 
\phantomsection\label{\detokenize{faq:id2}}{\hyperref[\detokenize{faq:how-do-we-setup-pqr}]{\sphinxcrossref{How do we setup pqr?}}}

\item {} 
\phantomsection\label{\detokenize{faq:id3}}{\hyperref[\detokenize{faq:is-abk-supported}]{\sphinxcrossref{Is abk supported?}}}

\end{itemize}
\end{sphinxShadowBox}


\section{What is abc?}
\label{\detokenize{faq:what-is-abc}}
Answer to What is abc….


\section{How do we setup pqr?}
\label{\detokenize{faq:how-do-we-setup-pqr}}
Detailed instructions on how to setup pqr….


\section{Is abk supported?}
\label{\detokenize{faq:is-abk-supported}}
Insights on backward compatibility and support matrix or {\hyperref[\detokenize{relnotes:doc-relnotes}]{\sphinxcrossref{\DUrole{std,std-ref}{links to the same}}}}.


\chapter{Indices and tables}
\label{\detokenize{index:indices-and-tables}}\begin{itemize}
\item {} 
\DUrole{xref,std,std-ref}{genindex}

\item {} 
\DUrole{xref,std,std-ref}{modindex}

\item {} 
\DUrole{xref,std,std-ref}{search}

\end{itemize}



\renewcommand{\indexname}{Index}
\printindex
\end{document}