%% Generated by Sphinx.
\def\sphinxdocclass{report}
\documentclass[letterpaper,10pt,english]{sphinxmanual}
\ifdefined\pdfpxdimen
   \let\sphinxpxdimen\pdfpxdimen\else\newdimen\sphinxpxdimen
\fi \sphinxpxdimen=.75bp\relax

\PassOptionsToPackage{warn}{textcomp}
\usepackage[utf8]{inputenc}
\ifdefined\DeclareUnicodeCharacter
% support both utf8 and utf8x syntaxes
\edef\sphinxdqmaybe{\ifdefined\DeclareUnicodeCharacterAsOptional\string"\fi}
  \DeclareUnicodeCharacter{\sphinxdqmaybe00A0}{\nobreakspace}
  \DeclareUnicodeCharacter{\sphinxdqmaybe2500}{\sphinxunichar{2500}}
  \DeclareUnicodeCharacter{\sphinxdqmaybe2502}{\sphinxunichar{2502}}
  \DeclareUnicodeCharacter{\sphinxdqmaybe2514}{\sphinxunichar{2514}}
  \DeclareUnicodeCharacter{\sphinxdqmaybe251C}{\sphinxunichar{251C}}
  \DeclareUnicodeCharacter{\sphinxdqmaybe2572}{\textbackslash}
\fi
\usepackage{cmap}
\usepackage[T1]{fontenc}
\usepackage{amsmath,amssymb,amstext}
\usepackage{babel}
\usepackage{times}
\usepackage[Bjarne]{fncychap}
\usepackage{sphinx}

\fvset{fontsize=\small}
\usepackage{geometry}

% Include hyperref last.
\usepackage{hyperref}
% Fix anchor placement for figures with captions.
\usepackage{hypcap}% it must be loaded after hyperref.
% Set up styles of URL: it should be placed after hyperref.
\urlstyle{same}

\addto\captionsenglish{\renewcommand{\figurename}{Fig.\@ }}
\makeatletter
\def\fnum@figure{\figurename\thefigure{}}
\makeatother
\addto\captionsenglish{\renewcommand{\tablename}{Table }}
\makeatletter
\def\fnum@table{\tablename\thetable{}}
\makeatother
\addto\captionsenglish{\renewcommand{\literalblockname}{Listing}}

\addto\captionsenglish{\renewcommand{\literalblockcontinuedname}{continued from previous page}}
\addto\captionsenglish{\renewcommand{\literalblockcontinuesname}{continues on next page}}
\addto\captionsenglish{\renewcommand{\sphinxnonalphabeticalgroupname}{Non-alphabetical}}
\addto\captionsenglish{\renewcommand{\sphinxsymbolsname}{Symbols}}
\addto\captionsenglish{\renewcommand{\sphinxnumbersname}{Numbers}}

\addto\extrasenglish{\def\pageautorefname{page}}




    \usepackage{sphinx}
    \makeatletter
    \newcommand{\sphinxbackoftitlepage}{%
    \vspace*{\fill}
    \begingroup
      \mbox{}
      \vfill    
      Pulse Secure, LLC
      \newline
      2700 Zanker Road, Suite 200
      \newline
      San Jose, CA 95134 
      \newline
      https://www.pulsesecure.net
      \vspace{2cm}
      \newline
      \vspace{3cm}
      © 2019 by Pulse Secure, LLC. All rights reserved

    \endgroup
    \vspace*{\fill}
    \clearpage
    }
    \makeatother
    

\title{PSdemo Developer's Guide}
\date{Oct 11, 2019}
\release{(alpha)}
\author{Pulse Secure, LLC}
\newcommand{\sphinxlogo}{\sphinxincludegraphics{psdemo_logo.png}\par}
\renewcommand{\releasename}{1.0}
\makeindex
\begin{document}

\pagestyle{empty}
\sphinxmaketitle
\pagestyle{plain}
\sphinxtableofcontents
\pagestyle{normal}
\phantomsection\label{\detokenize{dev-guide::doc}}


TBD

\begin{sphinxShadowBox}
\begin{itemize}
\item {} 
\phantomsection\label{\detokenize{dev-guide:id1}}{\hyperref[\detokenize{dev-guide:introduction}]{\sphinxcrossref{Introduction}}}

\item {} 
\phantomsection\label{\detokenize{dev-guide:id2}}{\hyperref[\detokenize{dev-guide:api-for-some-functionality}]{\sphinxcrossref{API for some functionality}}}
\begin{itemize}
\item {} 
\phantomsection\label{\detokenize{dev-guide:id3}}{\hyperref[\detokenize{dev-guide:pet-store-apis}]{\sphinxcrossref{Pet Store APIs}}}
\begin{itemize}
\item {} 
\phantomsection\label{\detokenize{dev-guide:id4}}{\hyperref[\detokenize{dev-guide:pet}]{\sphinxcrossref{pet}}}

\item {} 
\phantomsection\label{\detokenize{dev-guide:id5}}{\hyperref[\detokenize{dev-guide:store}]{\sphinxcrossref{store}}}

\item {} 
\phantomsection\label{\detokenize{dev-guide:id6}}{\hyperref[\detokenize{dev-guide:user}]{\sphinxcrossref{user}}}

\end{itemize}

\end{itemize}

\item {} 
\phantomsection\label{\detokenize{dev-guide:id7}}{\hyperref[\detokenize{dev-guide:example-code}]{\sphinxcrossref{Example Code}}}
\begin{itemize}
\item {} 
\phantomsection\label{\detokenize{dev-guide:id8}}{\hyperref[\detokenize{dev-guide:using-api-for-xyz-use-case}]{\sphinxcrossref{Using API for xyz use case}}}

\item {} 
\phantomsection\label{\detokenize{dev-guide:id9}}{\hyperref[\detokenize{dev-guide:using-api-for-mno-use-case}]{\sphinxcrossref{Using API for mno use case}}}

\end{itemize}

\end{itemize}
\end{sphinxShadowBox}


\bigskip\hrule\bigskip



\chapter{Introduction}
\label{\detokenize{dev-guide:introduction}}
TBD


\chapter{API for some functionality}
\label{\detokenize{dev-guide:api-for-some-functionality}}
TBD


\section{Pet Store APIs}
\label{\detokenize{dev-guide:pet-store-apis}}

\subsection{pet}
\label{\detokenize{dev-guide:pet}}
\begin{sphinxadmonition}{note}{POST /pet}

Add a new pet to the store\begin{itemize}
\item {} 
\sphinxstylestrong{Description: 
}

\item {} 
\sphinxstylestrong{Consumes: 
}{[}‘application/json’, ‘application/xml’{]}

\item {} 
\sphinxstylestrong{Produces: 
}{[}‘application/xml’, ‘application/json’{]}

\end{itemize}


\sphinxstylestrong{Parameters}


\begin{savenotes}\sphinxattablestart
\centering
\begin{tabulary}{\linewidth}[t]{|T|T|T|T|}
\hline
\sphinxstyletheadfamily 
Name
&\sphinxstyletheadfamily 
Position
&\sphinxstyletheadfamily 
Description
&\sphinxstyletheadfamily 
Type
\\
\hline
body
&
body
&
Pet object that needs to be added to the store
&

\\
\hline
\end{tabulary}
\par
\sphinxattableend\end{savenotes}

\sphinxstylestrong{Responses}

\sphinxstyleemphasis{405 - Invalid input}
\end{sphinxadmonition}

\begin{sphinxadmonition}{note}{PUT /pet}

Update an existing pet\begin{itemize}
\item {} 
\sphinxstylestrong{Description: 
}

\item {} 
\sphinxstylestrong{Consumes: 
}{[}‘application/json’, ‘application/xml’{]}

\item {} 
\sphinxstylestrong{Produces: 
}{[}‘application/xml’, ‘application/json’{]}

\end{itemize}


\sphinxstylestrong{Parameters}


\begin{savenotes}\sphinxattablestart
\centering
\begin{tabulary}{\linewidth}[t]{|T|T|T|T|}
\hline
\sphinxstyletheadfamily 
Name
&\sphinxstyletheadfamily 
Position
&\sphinxstyletheadfamily 
Description
&\sphinxstyletheadfamily 
Type
\\
\hline
body
&
body
&
Pet object that needs to be added to the store
&

\\
\hline
\end{tabulary}
\par
\sphinxattableend\end{savenotes}

\sphinxstylestrong{Responses}

\sphinxstyleemphasis{400 - Invalid ID supplied}

\sphinxstyleemphasis{404 - Pet not found}

\sphinxstyleemphasis{405 - Validation exception}
\end{sphinxadmonition}

\begin{sphinxadmonition}{note}{GET /pet/findByStatus}

Finds Pets by status\begin{itemize}
\item {} 
\sphinxstylestrong{Description: 
}Multiple status values can be provided with comma separated strings

\item {} 
\sphinxstylestrong{Produces: 
}{[}‘application/xml’, ‘application/json’{]}

\end{itemize}


\sphinxstylestrong{Parameters}


\begin{savenotes}\sphinxattablestart
\centering
\begin{tabulary}{\linewidth}[t]{|T|T|T|T|}
\hline
\sphinxstyletheadfamily 
Name
&\sphinxstyletheadfamily 
Position
&\sphinxstyletheadfamily 
Description
&\sphinxstyletheadfamily 
Type
\\
\hline
status
&
query
&
Status values that need to be considered for filter
&
array
\\
\hline
\end{tabulary}
\par
\sphinxattableend\end{savenotes}

\sphinxstylestrong{Responses}

\sphinxstyleemphasis{200 - successful operation}

\sphinxstyleemphasis{400 - Invalid status value}
\end{sphinxadmonition}

\begin{sphinxadmonition}{note}{GET /pet/findByTags}

Finds Pets by tags\begin{itemize}
\item {} 
\sphinxstylestrong{Description: 
}Multiple tags can be provided with comma separated strings. Use tag1, tag2, tag3 for testing.

\item {} 
\sphinxstylestrong{Produces: 
}{[}‘application/xml’, ‘application/json’{]}

\end{itemize}


\sphinxstylestrong{Parameters}


\begin{savenotes}\sphinxattablestart
\centering
\begin{tabulary}{\linewidth}[t]{|T|T|T|T|}
\hline
\sphinxstyletheadfamily 
Name
&\sphinxstyletheadfamily 
Position
&\sphinxstyletheadfamily 
Description
&\sphinxstyletheadfamily 
Type
\\
\hline
tags
&
query
&
Tags to filter by
&
array
\\
\hline
\end{tabulary}
\par
\sphinxattableend\end{savenotes}

\sphinxstylestrong{Responses}

\sphinxstyleemphasis{200 - successful operation}

\sphinxstyleemphasis{400 - Invalid tag value}
\end{sphinxadmonition}

\begin{sphinxadmonition}{note}{GET /pet/\{petId\}}

Find pet by ID\begin{itemize}
\item {} 
\sphinxstylestrong{Description: 
}Returns a single pet

\item {} 
\sphinxstylestrong{Produces: 
}{[}‘application/xml’, ‘application/json’{]}

\end{itemize}


\sphinxstylestrong{Parameters}


\begin{savenotes}\sphinxattablestart
\centering
\begin{tabulary}{\linewidth}[t]{|T|T|T|T|}
\hline
\sphinxstyletheadfamily 
Name
&\sphinxstyletheadfamily 
Position
&\sphinxstyletheadfamily 
Description
&\sphinxstyletheadfamily 
Type
\\
\hline
petId
&
path
&
ID of pet to return
&
integer
\\
\hline
\end{tabulary}
\par
\sphinxattableend\end{savenotes}

\sphinxstylestrong{Responses}

\sphinxstyleemphasis{200 - successful operation}

\sphinxstyleemphasis{400 - Invalid ID supplied}

\sphinxstyleemphasis{404 - Pet not found}
\end{sphinxadmonition}

\begin{sphinxadmonition}{note}{POST /pet/\{petId\}}

Updates a pet in the store with form data\begin{itemize}
\item {} 
\sphinxstylestrong{Description: 
}

\item {} 
\sphinxstylestrong{Consumes: 
}{[}‘application/x-www-form-urlencoded’{]}

\item {} 
\sphinxstylestrong{Produces: 
}{[}‘application/xml’, ‘application/json’{]}

\end{itemize}


\sphinxstylestrong{Parameters}


\begin{savenotes}\sphinxattablestart
\centering
\begin{tabulary}{\linewidth}[t]{|T|T|T|T|}
\hline
\sphinxstyletheadfamily 
Name
&\sphinxstyletheadfamily 
Position
&\sphinxstyletheadfamily 
Description
&\sphinxstyletheadfamily 
Type
\\
\hline
petId
&
path
&
ID of pet that needs to be updated
&
integer
\\
\hline
name
&
formData
&
Updated name of the pet
&
string
\\
\hline
status
&
formData
&
Updated status of the pet
&
string
\\
\hline
\end{tabulary}
\par
\sphinxattableend\end{savenotes}

\sphinxstylestrong{Responses}

\sphinxstyleemphasis{405 - Invalid input}
\end{sphinxadmonition}

\begin{sphinxadmonition}{note}{DELETE /pet/\{petId\}}

Deletes a pet\begin{itemize}
\item {} 
\sphinxstylestrong{Description: 
}

\item {} 
\sphinxstylestrong{Produces: 
}{[}‘application/xml’, ‘application/json’{]}

\end{itemize}


\sphinxstylestrong{Parameters}


\begin{savenotes}\sphinxattablestart
\centering
\begin{tabulary}{\linewidth}[t]{|T|T|T|T|}
\hline
\sphinxstyletheadfamily 
Name
&\sphinxstyletheadfamily 
Position
&\sphinxstyletheadfamily 
Description
&\sphinxstyletheadfamily 
Type
\\
\hline
api\_key
&
header
&

&
string
\\
\hline
petId
&
path
&
Pet id to delete
&
integer
\\
\hline
\end{tabulary}
\par
\sphinxattableend\end{savenotes}

\sphinxstylestrong{Responses}

\sphinxstyleemphasis{400 - Invalid ID supplied}

\sphinxstyleemphasis{404 - Pet not found}
\end{sphinxadmonition}

\begin{sphinxadmonition}{note}{POST /pet/\{petId\}/uploadImage}

uploads an image\begin{itemize}
\item {} 
\sphinxstylestrong{Description: 
}

\item {} 
\sphinxstylestrong{Consumes: 
}{[}‘multipart/form-data’{]}

\item {} 
\sphinxstylestrong{Produces: 
}{[}‘application/json’{]}

\end{itemize}


\sphinxstylestrong{Parameters}


\begin{savenotes}\sphinxattablestart
\centering
\begin{tabulary}{\linewidth}[t]{|T|T|T|T|}
\hline
\sphinxstyletheadfamily 
Name
&\sphinxstyletheadfamily 
Position
&\sphinxstyletheadfamily 
Description
&\sphinxstyletheadfamily 
Type
\\
\hline
petId
&
path
&
ID of pet to update
&
integer
\\
\hline
additionalMetadata
&
formData
&
Additional data to pass to server
&
string
\\
\hline
file
&
formData
&
file to upload
&
file
\\
\hline
\end{tabulary}
\par
\sphinxattableend\end{savenotes}

\sphinxstylestrong{Responses}

\sphinxstyleemphasis{200 - successful operation}
\end{sphinxadmonition}


\subsection{store}
\label{\detokenize{dev-guide:store}}
\begin{sphinxadmonition}{note}{GET /store/inventory}

Returns pet inventories by status\begin{itemize}
\item {} 
\sphinxstylestrong{Description: 
}Returns a map of status codes to quantities

\item {} 
\sphinxstylestrong{Produces: 
}{[}‘application/json’{]}

\end{itemize}


\sphinxstylestrong{Parameters}


\begin{savenotes}\sphinxattablestart
\centering
\begin{tabulary}{\linewidth}[t]{|T|T|T|T|}
\hline
\sphinxstyletheadfamily 
Name
&\sphinxstyletheadfamily 
Position
&\sphinxstyletheadfamily 
Description
&\sphinxstyletheadfamily 
Type
\\
\hline
\end{tabulary}
\par
\sphinxattableend\end{savenotes}

\sphinxstylestrong{Responses}

\sphinxstyleemphasis{200 - successful operation}
\end{sphinxadmonition}

\begin{sphinxadmonition}{note}{POST /store/order}

Place an order for a pet\begin{itemize}
\item {} 
\sphinxstylestrong{Description: 
}

\item {} 
\sphinxstylestrong{Produces: 
}{[}‘application/xml’, ‘application/json’{]}

\end{itemize}


\sphinxstylestrong{Parameters}


\begin{savenotes}\sphinxattablestart
\centering
\begin{tabulary}{\linewidth}[t]{|T|T|T|T|}
\hline
\sphinxstyletheadfamily 
Name
&\sphinxstyletheadfamily 
Position
&\sphinxstyletheadfamily 
Description
&\sphinxstyletheadfamily 
Type
\\
\hline
body
&
body
&
order placed for purchasing the pet
&

\\
\hline
\end{tabulary}
\par
\sphinxattableend\end{savenotes}

\sphinxstylestrong{Responses}

\sphinxstyleemphasis{200 - successful operation}

\sphinxstyleemphasis{400 - Invalid Order}
\end{sphinxadmonition}

\begin{sphinxadmonition}{note}{GET /store/order/\{orderId\}}

Find purchase order by ID\begin{itemize}
\item {} 
\sphinxstylestrong{Description: 
}For valid response try integer IDs with value \textgreater{}= 1 and \textless{}= 10. Other values will generated exceptions

\item {} 
\sphinxstylestrong{Produces: 
}{[}‘application/xml’, ‘application/json’{]}

\end{itemize}


\sphinxstylestrong{Parameters}


\begin{savenotes}\sphinxattablestart
\centering
\begin{tabulary}{\linewidth}[t]{|T|T|T|T|}
\hline
\sphinxstyletheadfamily 
Name
&\sphinxstyletheadfamily 
Position
&\sphinxstyletheadfamily 
Description
&\sphinxstyletheadfamily 
Type
\\
\hline
orderId
&
path
&
ID of pet that needs to be fetched
&
integer
\\
\hline
\end{tabulary}
\par
\sphinxattableend\end{savenotes}

\sphinxstylestrong{Responses}

\sphinxstyleemphasis{200 - successful operation}

\sphinxstyleemphasis{400 - Invalid ID supplied}

\sphinxstyleemphasis{404 - Order not found}
\end{sphinxadmonition}

\begin{sphinxadmonition}{note}{DELETE /store/order/\{orderId\}}

Delete purchase order by ID\begin{itemize}
\item {} 
\sphinxstylestrong{Description: 
}For valid response try integer IDs with positive integer value. Negative or non-integer values will generate API errors

\item {} 
\sphinxstylestrong{Produces: 
}{[}‘application/xml’, ‘application/json’{]}

\end{itemize}


\sphinxstylestrong{Parameters}


\begin{savenotes}\sphinxattablestart
\centering
\begin{tabulary}{\linewidth}[t]{|T|T|T|T|}
\hline
\sphinxstyletheadfamily 
Name
&\sphinxstyletheadfamily 
Position
&\sphinxstyletheadfamily 
Description
&\sphinxstyletheadfamily 
Type
\\
\hline
orderId
&
path
&
ID of the order that needs to be deleted
&
integer
\\
\hline
\end{tabulary}
\par
\sphinxattableend\end{savenotes}

\sphinxstylestrong{Responses}

\sphinxstyleemphasis{400 - Invalid ID supplied}

\sphinxstyleemphasis{404 - Order not found}
\end{sphinxadmonition}


\subsection{user}
\label{\detokenize{dev-guide:user}}
\begin{sphinxadmonition}{note}{POST /user}

Create user\begin{itemize}
\item {} 
\sphinxstylestrong{Description: 
}This can only be done by the logged in user.

\item {} 
\sphinxstylestrong{Produces: 
}{[}‘application/xml’, ‘application/json’{]}

\end{itemize}


\sphinxstylestrong{Parameters}


\begin{savenotes}\sphinxattablestart
\centering
\begin{tabulary}{\linewidth}[t]{|T|T|T|T|}
\hline
\sphinxstyletheadfamily 
Name
&\sphinxstyletheadfamily 
Position
&\sphinxstyletheadfamily 
Description
&\sphinxstyletheadfamily 
Type
\\
\hline
body
&
body
&
Created user object
&

\\
\hline
\end{tabulary}
\par
\sphinxattableend\end{savenotes}

\sphinxstylestrong{Responses}

\sphinxstyleemphasis{default - successful operation}
\end{sphinxadmonition}

\begin{sphinxadmonition}{note}{POST /user/createWithArray}

Creates list of users with given input array\begin{itemize}
\item {} 
\sphinxstylestrong{Description: 
}

\item {} 
\sphinxstylestrong{Produces: 
}{[}‘application/xml’, ‘application/json’{]}

\end{itemize}


\sphinxstylestrong{Parameters}


\begin{savenotes}\sphinxattablestart
\centering
\begin{tabulary}{\linewidth}[t]{|T|T|T|T|}
\hline
\sphinxstyletheadfamily 
Name
&\sphinxstyletheadfamily 
Position
&\sphinxstyletheadfamily 
Description
&\sphinxstyletheadfamily 
Type
\\
\hline
body
&
body
&
List of user object
&

\\
\hline
\end{tabulary}
\par
\sphinxattableend\end{savenotes}

\sphinxstylestrong{Responses}

\sphinxstyleemphasis{default - successful operation}
\end{sphinxadmonition}

\begin{sphinxadmonition}{note}{POST /user/createWithList}

Creates list of users with given input array\begin{itemize}
\item {} 
\sphinxstylestrong{Description: 
}

\item {} 
\sphinxstylestrong{Produces: 
}{[}‘application/xml’, ‘application/json’{]}

\end{itemize}


\sphinxstylestrong{Parameters}


\begin{savenotes}\sphinxattablestart
\centering
\begin{tabulary}{\linewidth}[t]{|T|T|T|T|}
\hline
\sphinxstyletheadfamily 
Name
&\sphinxstyletheadfamily 
Position
&\sphinxstyletheadfamily 
Description
&\sphinxstyletheadfamily 
Type
\\
\hline
body
&
body
&
List of user object
&

\\
\hline
\end{tabulary}
\par
\sphinxattableend\end{savenotes}

\sphinxstylestrong{Responses}

\sphinxstyleemphasis{default - successful operation}
\end{sphinxadmonition}

\begin{sphinxadmonition}{note}{GET /user/login}

Logs user into the system\begin{itemize}
\item {} 
\sphinxstylestrong{Description: 
}

\item {} 
\sphinxstylestrong{Produces: 
}{[}‘application/xml’, ‘application/json’{]}

\end{itemize}


\sphinxstylestrong{Parameters}


\begin{savenotes}\sphinxattablestart
\centering
\begin{tabulary}{\linewidth}[t]{|T|T|T|T|}
\hline
\sphinxstyletheadfamily 
Name
&\sphinxstyletheadfamily 
Position
&\sphinxstyletheadfamily 
Description
&\sphinxstyletheadfamily 
Type
\\
\hline
username
&
query
&
The user name for login
&
string
\\
\hline
password
&
query
&
The password for login in clear text
&
string
\\
\hline
\end{tabulary}
\par
\sphinxattableend\end{savenotes}

\sphinxstylestrong{Responses}

\sphinxstyleemphasis{200 - successful operation}

\sphinxstyleemphasis{400 - Invalid username/password supplied}
\end{sphinxadmonition}

\begin{sphinxadmonition}{note}{GET /user/logout}

Logs out current logged in user session\begin{itemize}
\item {} 
\sphinxstylestrong{Description: 
}

\item {} 
\sphinxstylestrong{Produces: 
}{[}‘application/xml’, ‘application/json’{]}

\end{itemize}


\sphinxstylestrong{Parameters}


\begin{savenotes}\sphinxattablestart
\centering
\begin{tabulary}{\linewidth}[t]{|T|T|T|T|}
\hline
\sphinxstyletheadfamily 
Name
&\sphinxstyletheadfamily 
Position
&\sphinxstyletheadfamily 
Description
&\sphinxstyletheadfamily 
Type
\\
\hline
\end{tabulary}
\par
\sphinxattableend\end{savenotes}

\sphinxstylestrong{Responses}

\sphinxstyleemphasis{default - successful operation}
\end{sphinxadmonition}

\begin{sphinxadmonition}{note}{GET /user/\{username\}}

Get user by user name\begin{itemize}
\item {} 
\sphinxstylestrong{Description: 
}

\item {} 
\sphinxstylestrong{Produces: 
}{[}‘application/xml’, ‘application/json’{]}

\end{itemize}


\sphinxstylestrong{Parameters}


\begin{savenotes}\sphinxattablestart
\centering
\begin{tabulary}{\linewidth}[t]{|T|T|T|T|}
\hline
\sphinxstyletheadfamily 
Name
&\sphinxstyletheadfamily 
Position
&\sphinxstyletheadfamily 
Description
&\sphinxstyletheadfamily 
Type
\\
\hline
username
&
path
&
The name that needs to be fetched. Use user1 for testing. 
&
string
\\
\hline
\end{tabulary}
\par
\sphinxattableend\end{savenotes}

\sphinxstylestrong{Responses}

\sphinxstyleemphasis{200 - successful operation}

\sphinxstyleemphasis{400 - Invalid username supplied}

\sphinxstyleemphasis{404 - User not found}
\end{sphinxadmonition}

\begin{sphinxadmonition}{note}{PUT /user/\{username\}}

Updated user\begin{itemize}
\item {} 
\sphinxstylestrong{Description: 
}This can only be done by the logged in user.

\item {} 
\sphinxstylestrong{Produces: 
}{[}‘application/xml’, ‘application/json’{]}

\end{itemize}


\sphinxstylestrong{Parameters}


\begin{savenotes}\sphinxattablestart
\centering
\begin{tabulary}{\linewidth}[t]{|T|T|T|T|}
\hline
\sphinxstyletheadfamily 
Name
&\sphinxstyletheadfamily 
Position
&\sphinxstyletheadfamily 
Description
&\sphinxstyletheadfamily 
Type
\\
\hline
username
&
path
&
name that need to be updated
&
string
\\
\hline
body
&
body
&
Updated user object
&

\\
\hline
\end{tabulary}
\par
\sphinxattableend\end{savenotes}

\sphinxstylestrong{Responses}

\sphinxstyleemphasis{400 - Invalid user supplied}

\sphinxstyleemphasis{404 - User not found}
\end{sphinxadmonition}

\begin{sphinxadmonition}{note}{DELETE /user/\{username\}}

Delete user\begin{itemize}
\item {} 
\sphinxstylestrong{Description: 
}This can only be done by the logged in user.

\item {} 
\sphinxstylestrong{Produces: 
}{[}‘application/xml’, ‘application/json’{]}

\end{itemize}


\sphinxstylestrong{Parameters}


\begin{savenotes}\sphinxattablestart
\centering
\begin{tabulary}{\linewidth}[t]{|T|T|T|T|}
\hline
\sphinxstyletheadfamily 
Name
&\sphinxstyletheadfamily 
Position
&\sphinxstyletheadfamily 
Description
&\sphinxstyletheadfamily 
Type
\\
\hline
username
&
path
&
The name that needs to be deleted
&
string
\\
\hline
\end{tabulary}
\par
\sphinxattableend\end{savenotes}

\sphinxstylestrong{Responses}

\sphinxstyleemphasis{400 - Invalid username supplied}

\sphinxstyleemphasis{404 - User not found}
\end{sphinxadmonition}


\chapter{Example Code}
\label{\detokenize{dev-guide:example-code}}
Perform the setup using the following commands

\begin{sphinxVerbatim}[commandchars=\\\{\}]
\PYG{n}{bash} \PYG{n}{xyz}\PYG{o}{.}\PYG{n}{py} \PYG{o}{\PYGZhy{}}\PYG{n}{o} \PYG{o}{\PYGZlt{}}\PYG{n}{outputfile}\PYG{o}{\PYGZgt{}} \PYG{o}{\PYGZhy{}}\PYG{n}{d} \PYG{o}{\PYGZlt{}}\PYG{n}{inputdir}\PYG{o}{\PYGZgt{}} \PYG{o}{\PYGZhy{}}\PYG{n}{t} \PYG{o}{\PYGZlt{}}\PYG{n+nb}{type}\PYG{o}{\PYGZgt{}}
\end{sphinxVerbatim}


\section{Using API for xyz use case}
\label{\detokenize{dev-guide:using-api-for-xyz-use-case}}

\section{Using API for mno use case}
\label{\detokenize{dev-guide:using-api-for-mno-use-case}}


\renewcommand{\indexname}{Index}
\printindex
\end{document}